\documentclass[12pt,a4paper]{article}
\usepackage[utf8]{inputenc}
\usepackage{amsmath,enumitem,amsfonts,amssymb,graphicx}
\usepackage{multicol}
\usepackage{sectsty}
\usepackage{commath}
\usepackage{tikz}
\usepackage{tikz-qtree}

\usetikzlibrary{trees}

\usepackage[%
    left=1.0in,%
    right=1.0in,%
    top=0.8in,%
    bottom=1.0in,%
]{geometry}%

\sectionfont{\fontsize{14.4}{12}\selectfont}
\title{\textbf{Principles of AI Planning
		\\{\Large Exercise Sheet 4}}}
\date{22.11.2019}

\makeatletter
\renewcommand{\@maketitle}
{
	\newpage
	\null
	\vskip 2em%
	\begin{center}%
		{\LARGE \@title \\ \par}%
	\end{center}%
	\par
} \makeatother


\begin{document}
	\begin{flushleft}
		Authors:\\
		Erick Rosete Beas | er165@uni-freiburg.de\\
		Jessica Lizeth Borja Diaz | jb986@uni-freiburg.de\\
	\end{flushleft}
	{\let\newpage\relax\maketitle}
	\begin{center} 
		\large 22.11.2019 
	\end{center}

\hfill\break
\section*{Exercise 4.1: Example for general regression}
	\textbf{a) Consider the STRIPS planning tasks, Solve this problem with a breadth-first search.}
	\\for simplicity we call the atoms the following way:

\begin{multicols}{2}
	\begin{center}
		\begin{tabular}{ c|c }
		atoms & name \\
		\hline
		juliet-dancing & jd \\
		juliet-at-home & jh \\    
		romeo-at-home & rh \\   
		romeo-dancing & rd \\
		romeo-at-work & rw \\ 
		\end{tabular}
	\end{center}

	\begin{flalign*}
	 	 & A=\{romeo,juliet\}
		\\ & I=\{rh\mapsto 1, jh \mapsto 1 \}
		\\ & \gamma = jd \land rh
		\\ & O =\{o_1, o_2, o_3\}
	\end{flalign*} 
\end{multicols}{}

\hfill\break
\begin{center}
	\begin{tabular}{ c | c }
	 	Go-work: &
		$ o_1= \langle rh, rw \land \neg rh \rangle $ \\[1em]
		\hline
		Go-home: &
		$ o_2= \langle rw, rh \land \neg rw \rangle $ \\[1em]
		\hline
		Go-dancing: &
		$ o_3= \langle jh, jd \land \neg jh \land 
				(rh \triangleright (rd \land \neg rh) )\rangle $'
	\end{tabular}{}
\end{center}

%%%%%%%%%%%%%%%%%%%%%% Regression equations %%%%%%%%%%%%%%%%%%%%%%%%%%%%%
\textbf{Operations: }

\begin{equation*}
	\begin {split}
	regr_{o_3}(jd \land rh) = jh \land (\top \lor ( jd \land \neg \bot))\\
					\land (\bot \lor ( rh \land \neg rh))\\
					= jh \land \top \land \bot = \bot\\
	\end {split}
\end{equation*}
\[	regr_{o_2} (jd \land rh)= rw \land jd \]
\begin{equation*}
	\begin {split}
	regr_{o_3}(rw \land jd) = jh \land (\top \lor ( jd \land \neg \bot))\\
							\land (\bot \lor ( rw \land \neg \bot))\\
							= jh \land rw
	\end {split}
\end{equation*}

%%%%%%%%%%%%%%%%%%%%%% Tree %%%%%%%%%%%%%%%%%%%%%%%%%%%%%
\hfill\break
\begin{center}
	\tikzset{
		treenode/.style = {align=left,text centered},
		env/.style      = {treenode, font=\large},
		sub/.style      = {treenode, font=\normalsize},
		accepted/.style      = {treenode, font=\normalsize, circle, draw=green},
		}
	\begin{tikzpicture}
		[	sloped,
			sibling distance        = 5em,
			level distance          = 8em,
			grow=right,
			edge from parent/.style = {draw, semithick, <-},
			edge from parent path=
			{(\tikzparentnode) -- (\tikzchildnode.west)}
		]
	\node [] {$jd \land rh$}
		child{
			node[label=below: \small{makes $rh$ false}]
			{$ \bot $}    
			edge from parent node[above] {$o_1$}
		}
		child{
			node []{$ rw \land jd $}
			child {
				node [sub,  label=right: 
				\small{Stop exploring, duplicate}] 
				{$rh \land jd$}
				edge from parent node [above] {$o_1$} 
			}
			child {
				node [sub,  label=below: 
				\small{makes $rw$ false}] 
				{$\bot$}
				edge from parent node [above] {$o_2$} 
			}
			child {
				node [sub] 
				{$ jh \land rw $}
					child {
						node [accepted,  label=right: 
						\small{goal state}] 
						{$jh \land rh$}
						edge from parent node [above] {$o_1$} 
					}
					child {
						node [sub,  label=right: 
						\small{makes $rw$ false}] 
						{$\bot$}
						edge from parent node [above] {$o_2$} 
					}
					child {
						node [sub,  label=right: 
						\small{makes $jh$ false}] 
						{$\bot$}
						edge from parent node [above] {$o_3$} 
					}
				edge from parent node [above] {$o_3$} 
			}
			edge from parent node[above] {$o_2$}
		}
		child{
			node[] {$\bot$} 
			edge from parent node[above]{$o_3$}
		};
	\end{tikzpicture}
\end{center}
Then the plan would be : $o_1, o_3, o_2$. This means: romeo goes to work, 
then juliet goes dancing but as romeo is at ``work"
he cannot go, finaly romeo returns happily to home.
%%%%%%%%%%%%%%%%%%%%%% Exercise 4.2 %%%%%%%%%%%%%%%%%%%%%%%%%%%%%
\section*{Exercise 4.2 (Exponential plan length)}
Show that for all $n \in N $ there is a conditional effect free planning task $\prod = \langle A, I, O, \gamma \rangle$ with $\abs{A} = \abs{O} = O(n)$ such that $ \abs{\pi} = O(2^{\abs{n}})$ where $\abs{\pi}$ is the length of a plan.\\\\
To prove it we will construct a planning task that works $\forall n \in N$, the task will be modeled as a binary counter, this counter will start in $0$ and go to $2^n - 1$, and as we know in a binary counter you must pass from every state until reaching the goal state, therefore the optimal plan will have a length $\pi = 2^n$.\\
\[A = \lbrace A_i, A_{i-1}, ..., A_1, A_0 \rbrace\]  
Where $A_i$ is the most significant bit, and $A_0$ is the least significant bit.\\
\[I = \lbrace A_i \mapsto 0, A_{i-1} \mapsto 0, ..., A_1 \mapsto 0, A_0 \mapsto 0 \rbrace\]  
The counter initial state starts with all bits in 0.
\[\gamma = \lbrace A_i \mapsto 1, A_{i-1} \mapsto 1, ..., A_1 \mapsto 1, A_0 \mapsto 1 \rbrace\] 
The goal will be all bits set to 1 as this represents the number $2^n - 1$.\\

\[O = \lbrace O_0, O_1, ..., O_{i-1}, O_{i} \rbrace \]
\[O_0 = \langle \neg A_0, A_0\rangle \]
\[O_1 = \langle \neg A_1 \land A_0, A_1 \land \neg A_0 \rangle \]
\[O_2 = \langle \neg A_2 \land A_1 \land A_0, A_2 \land \neg A_1 \land \neg A_0 \rangle \]
\begin{center}
	...
\end{center}
\[O_i = \langle \neg A_i \land A_{i-1} \land ... \land A_1 \land A_0, A_i\land \neg A_{i-1} \land ... \land \neg A_1 \land \neg A_0 \rangle \]
The operators are the same size as the amount of variables and don't have conditional effects, which satisfies the problems conditions.
Furthermore the operators flip all bits to the right to 0, and change the current bit to 1, which is what happens in a normal binary counter.\\
Describing the solution is similar to describing a recursive solution, we know that to set $A_j$ to 1 we must use the operator $O_j$, then when $A_j$ is turned to 1, the atoms/bits to the right are set to 0, and they must be set to 1 to reach the goal, to do these we must repeat the previous operators sequence, when they are all set to 1, then, $O_{j+1}$ is the only operator than can be used and again all bits to the right are empty and we must repeat this process until setting all atoms to 1 and reaching the goal. Therefore this task work as the desired binary counter, always having an exponential length to reach the goal $\forall n \in N$. Q.E.D.

\end{document}