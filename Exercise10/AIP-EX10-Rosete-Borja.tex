\documentclass[11pt,a4paper]{article}
\usepackage[utf8]{inputenc}
\usepackage{amsmath,enumitem,amsfonts,amssymb,graphicx,commath}
\usepackage{sectsty}
\usepackage{multicol}
\usepackage{tikz}

\graphicspath{ {./img/} }
\DeclareMathAlphabet{\pazocal}{OMS}{zplm}{m}{n}
\usetikzlibrary{arrows,automata}

\usepackage[%
    left=1in,%
    right=1.0in,%
    top=0.8in,%
    bottom=1in,%
]{geometry}%

\sectionfont
{\fontsize{14.4}{12}\selectfont}
\title{\textbf{Principles of AI Planning
		\\{\Large Exercise Sheet 10}}}
\makeatletter
\renewcommand{\@maketitle}
{
	\newpage
	\null
	\vskip 2em%
	\begin{center}%
		{\LARGE \@title \\ \par}%
	\end{center}%
	\par
} \makeatother

\begin{document}
\begin{flushleft}
	Authors:\\
	Erick Rosete Beas | er165@uni-freiburg.de\\
	Jessica Lizeth Borja Diaz | jb986@uni-freiburg.de\\
\end{flushleft}
{\let\newpage\relax\maketitle}
\begin{center} 
	\large 17.01.2020
\end{center}


%%%%%%%%%%%%%%%%%%%%%  Ejercicio 1 %%%%%%%%%%%%%%%%%%%%%%%%%
\section*{Exercise 10.1 - Affecting labels vs. orthogonality}
Prove the following: Let $\pazocal{A}_i$ be an abstraction of some transition system $\pazocal{T}$ with abstraction mapping $\alpha_i$ for $i \in \{1, 2\}$. If no label of $\pazocal{T}$ affects both $\pazocal{A}_1$ and $\pazocal{A}_2$, then $\alpha_1$ and $\alpha_2$ are orthogonal.\\
Definition of orthogonality:\\
Let $\alpha_1$ and $\alpha_2$ be abstraction mappings on $\pazocal{T}$.
We say that $\alpha_1$ and $\alpha_2$ are orthogonal if for all transitions
$\langle s, l, t \rangle$ of $\pazocal{T}$ , we have $\alpha_i(s) \neq \alpha_i(t)$  for at most one $i \in \{1,2\}$.\\
Definition of affecting transition labels.\\
For a transition system $\pazocal{A}$ and a label $l$ of $\pazocal{A}$, we say that $l$ affects $\pazocal{A}$ if $\pazocal{A}$ has a transition $\langle s, l, t \rangle$ with $s \neq t$.\\
\pagebreak
%%%%%%%%%%%%%%%%%%%%%  Ejercicio 2 %%%%%%%%%%%%%%%%%%%%%%%%%
\section*{Exercise 10.2 - Potential heuristics: consistency constraints}
Let $\Pi = \langle V, I, O, \gamma \rangle$ be an $SAS^+$ planning task in transition normal form, and let $\pazocal{F} = \{f_{v=d} \mid v \in V, d \in \pazocal{D}_v\}$ be the set of all atomic features over $\Pi$. Finally, let:
\[h(s) = \sum_{fact\:v=d} w_{v=d} \cdot f_{v=d}(s) \]
be the potential heuristic with potentials $w_{v=d}$ for all $v \in V , d \in \pazocal{D}_v$, such that for all $o \in O,$ the following constraint is satisfied:
\[\sum_{fact\:v=d \text{ consumed by } o} w_{v=d} - \sum_{fact\:v=d \text{ produced by } o} w_{v=d} \leq cost(o)\]
Prove: Then $h$ is consistent, i. e., $h(s) - h(t) \leq cost(o)$ for all transitions $(s, o, t)$ in $\pazocal{T}(\Pi)$.
\end{document}