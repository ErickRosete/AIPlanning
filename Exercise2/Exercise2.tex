\documentclass[12pt,a4paper]{article}
\usepackage[utf8]{inputenc}
\usepackage{amsmath,multicol}
\usepackage{amsfonts}
\usepackage{amssymb}
\usepackage{graphicx}
\usepackage{enumitem}

\usepackage[%
    left=1.0in,%
    right=1.0in,%
    top=0.8in,%
    bottom=1.0in,%
]{geometry}%

\title{\textbf{Principles of AI Planning
		\\{\Large Exercise Sheet 2}}}
\date{6.11.2019}

\makeatletter
\renewcommand{\@maketitle}
{
	\newpage
	\null
	\vskip 2em%
	\begin{center}%
		{\LARGE \@title \\ \par}%
	\end{center}%
	\par
} \makeatother

\begin{document}
	\begin{flushleft}
		Authors:\\
		Erick Rosete Beas | er165@uni-freiburg.de\\
		Jessica Lizeth Borja Diaz | jb986@uni-freiburg.de\\
	\end{flushleft}
	{\let\newpage\relax\maketitle}
	\begin{center} 
		\large 6.11.2019 
	\end{center}

	\section*{Exercise 2.1 - Effect Normal Form}
	\textbf{a) Transform the operator into ENF}
	
	\begin{align*} 
		\langle\neg e \lor f,
		\\ (a \triangleright (b \triangleright c)) \land
		\\ (\neg d \triangleright c) \land
		\\ (\neg(\neg c \land \neg a) \triangleright (d \land \neg e)) \land 
		\\ (d \triangleright \neg e) \rangle 
	\end{align*} 

\hfill\break
\setlength{\columnsep}{50pt}
\begin{multicols}{2}
	De Morgan's Law
	\begin{align*} 
		\langle\neg e \lor f, 
		\\ (a \triangleright (b \triangleright c))  \land 
		\\ (\neg d \triangleright c) \land 
		\\ ((c \lor a) \triangleright (d \land \neg e)) \land 
		\\ (d \triangleright \neg e) \rangle
	\end{align*} 

	(7)
	\begin{align*} 
		\langle\neg e \lor f, 
		\\ ((a \land b) \triangleright c))  \land 
		\\ (\neg d \triangleright c) \land 
		\\ ((c \lor a) \triangleright (d \land \neg e)) \land 
		\\ (d \triangleright \neg e) \rangle
	\end{align*} 
	
	(9)
	\begin{align*} 
		\langle\neg e \lor f, 
		\\ ((a \land b \lor \neg d) \triangleright c) \land 
		\\ ((c \lor a) \triangleright (d \land \neg e)) \land 
		\\ (d \triangleright \neg e) \rangle
	\end{align*} 
	
	(8)
	\begin{align*} 
		\langle\neg e \lor f,
		\\ ((a \land b \lor \neg d) \triangleright c) \land 
		\\ ((c \lor a) \triangleright d) \land 
		\\((c \lor a) \triangleright \neg e) \land 
		\\(d \triangleright \neg e) \rangle
	\end{align*} 

	(9)
	\begin{align*} 
		\langle\neg e \lor f, 
		\\ ((a \land b \lor \neg d) \triangleright c) \land 
		\\ ((c \lor a) \triangleright d) \land 
		\\ ((c \lor a \lor d) \triangleright \neg e) \rangle
	\end{align*}
\end{multicols}

	\hfill\break
	\textbf{b) Transform the operator into positive normal form}
	\begin{align*} 
		\langle\neg e \lor f, 
		\\ ((a \land b \lor \neg d) \triangleright c) \land 
		\\ ((c \lor a) \triangleright d) \land 
		\\ ((c \lor a \lor d) \triangleright \neg e) \rangle
	\end{align*} 

\hfill\break
\setlength{\columnsep}{50pt}
\begin{multicols}{2}
\noindent First we identify the negative atom $\neg e$ and we change it for $\hat{e}$
	\begin{align*} 
		\langle \hat{e} \lor f, 
		\\ ((a \land b \lor \neg d) \triangleright c) \land 
		\\ ((c \lor a) \triangleright d) \land 
		\\ ((c \lor a \lor d) \triangleright \neg e) \rangle
	\end{align*} 
	We change effect $\neg e$ for $\neg{e} \land \hat{e}$
	\begin{align*} 
		\langle \hat{e} \lor f, 
		\\ ((a \land b \lor \neg d) \triangleright c) \land 
		\\ ((c \lor a) \triangleright d) \land 
		\\ ((c \lor a \lor d) \triangleright (\neg{e} \land \hat{e})) \rangle
	\end{align*} 
	We identify the negative atom $\neg d$ and we change it for $\hat{d}$.
	\begin{align*} 
		\langle \hat{e} \lor f, 
		\\ ((a \land b \lor \hat{d}) \triangleright c) \land 
		\\ ((c \lor a) \triangleright d) \land 
		\\ ((c \lor a \lor d) \triangleright (\neg{e} \land \hat{e})) \rangle
	\end{align*} 
	We change effect $d$ for $d \land \neg \hat{d}$
	\begin{align*} 
		\langle \hat{e} \lor f, 
		\\ ((a \land b \lor \hat{d}) \triangleright c) \land 
		\\ ((c \lor a) \triangleright (d \land \neg \hat{d})) \land 
		\\ ((c \lor a \lor d) \triangleright (\neg{e} \land \hat{e})) \rangle
	\end{align*} 		
	(8)
	\begin{align*} 
		\langle \hat{e} \lor f, 
		\\ ((a \land b \lor \hat{d}) \triangleright c) \land 
		\\ ((c \lor a) \triangleright d) \land 
		\\ ((c \lor a) \triangleright \neg\hat{d}) \land 
		\\ ((c \lor a \lor d) \triangleright (\neg{e} \land \hat{e}) \rangle
	\end{align*} 
	(8)
	\begin{align*} 
		\langle \hat{e} \lor f, 
		\\ ((a \land b \lor \hat{d}) \triangleright c) \land 
		\\ ((c \lor a) \triangleright d) \land 
		\\ ((c \lor a) \triangleright \neg\hat{d}) \land 
		\\ ((c \lor a \lor d) \triangleright \neg{e}) \land 
		\\ ((c \lor a \lor d) \triangleright \hat{e}) \rangle
	\end{align*} 
\end{multicols}

\section{Exercise 2.2 - PDDL set cover}

\begin{enumerate}[label=\alph*.,start=3]
	\item With the integrated planner the sets selected are $S1, S2, S3$, 
		this is a satisficing plan, however not optimal. In contrast the 
		custom planner selects the sets $S4,S5$ which corresponds to the optimal plan.
	\item Arbitrary set cover selects multiple sets to entirely cover the universe set, 
		but it does not check if it is the optimal plan although it satisfies the goal,
		making it a satisficing plan approach. On the other hand, cardinality minimal set cover
		refers to selecting the minimal amount of sets to cover the entire universe,
		this is what an optimal plan does. If no set combination exists such that the 
		whole universe is covered by its elements, there cannot be a plan.
\end{enumerate}
\end{document}