\documentclass[12pt,a4paper]{article}
\usepackage[utf8]{inputenc}
\usepackage{amsmath,enumitem,amsfonts,amssymb,graphicx}
\usepackage{multicol}
\usepackage{sectsty}

\usepackage{tikz}
\usetikzlibrary{trees}

\usepackage[%
    left=1.0in,%
    right=1.0in,%
    top=0.8in,%
    bottom=1.0in,%
]{geometry}%

\sectionfont{\fontsize{14.4}{12}\selectfont}
\title{\textbf{Principles of AI Planning
		\\{\Large Exercise Sheet 3}}}
\date{15.11.2019}

\makeatletter
\renewcommand{\@maketitle}
{
	\newpage
	\null
	\vskip 2em%
	\begin{center}%
		{\LARGE \@title \\ \par}%
	\end{center}%
	\par
} \makeatother


\begin{document}
	\begin{flushleft}
		Authors:\\
		Erick Rosete Beas | er165@uni-freiburg.de\\
		Jessica Lizeth Borja Diaz | jb986@uni-freiburg.de\\
	\end{flushleft}
	{\let\newpage\relax\maketitle}
	\begin{center} 
		\large 6.11.2019 
	\end{center}

\hfill\break
\section*{Exercise 3.1: STRIPS Regression}
	\textbf{a) Consider the STRIPS planning tasks with:}
	\begin{flalign*}
	 	 & A=\{a,b,c,d,e\}
		\\ & I=\{a\mapsto 0, b\mapsto 1 , c\mapsto 0, d\mapsto 1,e\mapsto 1 \}
		\\ & \gamma = a \land d
		\\ & O =\{o_1, o_2, o_3\}
		\\ & o_1=\langle b\land d, c\land e \land \neg d \rangle
		\\ & o_2=\langle b, a\land \neg c \land \neg d \rangle
		\\ & o_3=\langle a, d \rangle
	\end{flalign*} 
	Solve this problem with a breadth-first search using the STRIPS regression method.
\hfill\break
\tikzset{
  treenode/.style = {align=left,text centered},
  env/.style      = {treenode, font=\large},
  sub/.style      = {treenode, font=\normalsize},
  end/.style      = {treenode, font=\normalsize, circle, draw=green},
}
\begin{tikzpicture}
  [
    grow                    = right,
    sibling distance        = 5em,
    level distance          = 9em,
    edge from parent/.style = {draw, latex-},
    every node/.style       = {font=\normalsize},
    sloped
  ]
  \node [env] {$\gamma = a \land d$}
	child { node [env] {$a$}
			child { node [sub,  label=right: \small{Stop exploring, duplicate}] 
						{$a$}
					edge from parent node [above] {$regr_{o_3}$} 
			}
			child { node [end, label=right:\small{This represents an allowed initial state}] {$b$}
					edge from parent node [above] {$regr_{o_2}$}
			}
			child { node [sub, label=right: 
					\small{Stop exploring, $regr_{o_1}(a)$ is subset of a}] 
					{$b \land d \land a $}
					edge from parent node [above] {$regr_{o_1}$}
			}
		edge from parent node [above] {$regr_{o_3}$}
	}
	child { node [env] {$\bot$}
		edge from parent node [above] {$regr_{o_2}$} 
	}
	child { node [env] {$\bot$}
		edge from parent node [above] {$regr_{o_1}$} 
	};
    
\end{tikzpicture}

\newpage
\section*{Exercise 3.2: Problem modeling}
The plan obtained by the custom planner with the PDDL program was the following:
\begin{center}
	\begin{tabular}{l}
	(move $g$ $b$)\\
	(sample-rock $b$ $r_1$)\\
	(move $b$ $c$)\\
	(sample-rock $c$ $r_2$)\\
	(move $c$ $d$)\\
	(sample-rock $d$ $r_3$)\\
	(move $d$ $e$)\\
	(sample-rock $e$ $r_4$)\\
	(move $e$ $g$)\\
	(move $g$ $f$)\\
	(move $f$ $a$)\\
	(transmit-data $a$ $r_4$)\\
	(transmit-data $a$ $r_3$)\\
	(transmit-data $a$ $r_2$)\\
	(transmit-data $a$ $r_1$)\\
	\end{tabular}
\end{center}
We can notice the obtained plan with the custom planner is optimal as it satisfies the problem conditions with the least cost.

\end{document}