\documentclass[12pt,a4paper]{article}
\usepackage[utf8]{inputenc}
\usepackage{amsmath}
\usepackage{amsfonts}
\usepackage{amssymb}
\usepackage{graphicx}
\usepackage{enumitem}

\title{\textbf{Principles of AI Planning
\\{\Large Exercise Sheet 1}}}
\date{1.11.2019}

\makeatletter
\renewcommand{\@maketitle}
{
\newpage
 \null
 \vskip 2em%
 \begin{center}%
  {\LARGE \@title \\ \par}%
 \end{center}%
 \par
 } \makeatother

\begin{document}
	\begin{flushleft}
		Authors:\\
		Erick Rosete Beas | er165@uni-freiburg.de\\
		Jessica Lizeth Borja Diaz | jb986@uni-freiburg.de\\
	\end{flushleft}
	{\let\newpage\relax\maketitle}
	\begin{center} 
		\large 1.11.2019 
	\end{center}
	
	\section*{Exercise 1.2 - State Space Size}
	The goal is to measure the time required to generate the whole state space if every state generation if generating one state
	take $1x10^{-6} s$.\\\\
	Lets summarize the important data and variables:
	\begin{itemize}[noitemsep]
		\item Five robotic vacuum cleaner.
		\item 10x10 discrete cells.
		\item Each robot is in exactly one cell and several robots can be in the same cell at the same time.
		\item Each cell is either clean or dirty.
		\item Each robot has a battery with 20 charge levels
	\end{itemize}
	The whole state space will be of the following size:
	\[
		\underbrace{(100^{5})}_{\text{robot positions}}*
		\underbrace{(20^{5})}_{\text{charging states}}*
		\underbrace{(2^{100})}_{\text{cell state (clean or dirty)}} = 4.056*10^{46}
	\]
	Creating the whole state space will take
	\[4.056*10^{46}*10^{-6}s = 4.056*10^{40} s = 1.286*10^{33} years\]
	Which is far bigger than the age of the universe.
	
	
	\section*{Exercise 1.3 - Planning literature}
	The questions that we will like to discuss during the course of the lecture are the following:
	\begin{enumerate}
		\item How can we generate an heuristic function of a planning task?
		\item During the article it was mentioned that the current model-based approaches like SAT may spend a lot of time solving problems with a big state space size. How does the recent development \textbf{SAT-based planning} overcomes this issue? Will this type of planner be included in the lecture?
		\item Is there any other model-based approach besides SAT that combined with heuristic functions performs equal or better than SAT-based planning?
		\item If not, why is SAT combined with heuristics instead of other model-based approaches?
		\item Additionally from Torchlight, has there been any new advances on the field of automatically generating an optimal heuristic function?
		\item Is LM-cut heuristic still the state of the art in planners? Will this topic be covered in the course?
	\end{enumerate}

	
\end{document}