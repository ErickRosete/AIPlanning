\documentclass[12pt,a4paper]{article}
\usepackage[utf8]{inputenc}
\usepackage{amsmath}
\usepackage{amsfonts}
\usepackage{amssymb}
\usepackage{graphicx}
\usepackage[width=18.00cm, height=25.00cm]{geometry}
\begin{document}
	\section*{Exercise Sheet 6}
	\subsection*{Exercise 6.1 - Delete Relaxation}
	a) Give the relaxation $\Pi^+$ of $\Pi$
	\[A = \lbrace haveCake, eatenCake, haveNoCake \rbrace \]
	\[I = \lbrace haveCake \rightarrow 0, eatenCake \rightarrow 0, haveNoCake \rightarrow 1 \rbrace \]
	\[ O^+ = \lbrace eatCake, bakeCake \rbrace \]
	\[eatCake^+ = \langle  haveCake, haveNoCake \land eatenCake \rangle \]
	\[bakeCake^+ = \langle  haveNoCake, haveCake \rangle \]
	\[\gamma = haveCake \land eatenCake \]
	The negative effects of the operators were removed.\\\\
	b) Give a sequence of $\pi$ of operators (as short as possible) from O such that $\pi$ is not a plan of $\Pi$ but $\pi^+$ is a plan of $\Pi^+$.\\
	\[ \pi^+ = bakeCake^+, eatCake^+ \]
	Why? We can simulate this plan in both planning tasks to demonstrate it.\\
	States after $bakeCake^+$ 	
	\[S_1 = \lbrace haveCake \rightarrow 1, eatenCake \rightarrow 0, haveNoCake \rightarrow 1 \rbrace \]
	States after $eatCake^+$ 	
	\[S_1 = \lbrace haveCake \rightarrow 1, eatenCake \rightarrow 1, haveNoCake \rightarrow 1 \rbrace \]
	Goal accomplished. \\\\
	If we run the same plan $\pi$ in $\Pi$\\
	States after $bakeCake$ 	
	\[S_1 = \lbrace haveCake \rightarrow 1, eatenCake \rightarrow 0, haveNoCake \rightarrow 0 \rbrace \]
	States after $eatCake$ 	
	\[S_1 = \lbrace haveCake \rightarrow 0, eatenCake \rightarrow 1, haveNoCake \rightarrow 1 \rbrace \]
	Goal is not accomplished as we don't have cake.
\end{document}