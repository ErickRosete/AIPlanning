\documentclass[12pt,a4paper]{article}
\usepackage[utf8]{inputenc}
\usepackage{amsmath,enumitem,amsfonts,amssymb,graphicx,commath}
\usepackage[width=18.00cm, height=25.00cm]{geometry}
\usepackage{sectsty}

\sectionfont
{\fontsize{14.4}{12}\selectfont}
\title{\textbf{Principles of AI Planning
		\\{\Large Exercise Sheet 6}}}
\date{06.12.2019}

\makeatletter
\renewcommand{\@maketitle}
{
	\newpage
	\null
	\vskip 2em%
	\begin{center}%
		{\LARGE \@title \\ \par}%
	\end{center}%
	\par
} \makeatother

\begin{document}
\begin{flushleft}
	Authors:\\
	Erick Rosete Beas | er165@uni-freiburg.de\\
	Jessica Lizeth Borja Diaz | jb986@uni-freiburg.de\\
\end{flushleft}
{\let\newpage\relax\maketitle}
\begin{center} 
	\large 06.12.2019 
\end{center}

	\section*{Exercise Sheet 6}
	\subsection*{Exercise 6.1 - Delete Relaxation}
	\begin{enumerate}[label=(\alph*)]
		\item \textbf{Give the relaxation $\Pi^+$ of $\Pi$}
		
		\begin{center}
			\begin{tabular}{l}
				$A = \lbrace haveCake, eatenCake, haveNoCake \rbrace $ \\
				$I = \lbrace haveCake \rightarrow 0, eatenCake \rightarrow 0, haveNoCake \rightarrow 1 \rbrace $ \\
				$ O^+ = \lbrace eatCake^+, bakeCake^+ \rbrace $ \\
				$eatCake^+ = \langle  haveCake, haveNoCake \land eatenCake \rangle $\\
				$bakeCake^+ = \langle  haveNoCake, haveCake \rangle $\\
				$\gamma = haveCake \land eatenCake $
			\end{tabular}
		\end{center}
	The negative effects of the operators were removed.\\\\
	
	\item \textbf{Give a sequence of $\pi$ of operators (as short as possible) from O such that $\pi$ is not a plan of $\Pi$ but $\pi^+$ is a plan of $\Pi^+$.}\\
		\[ \pi^+ = \{ bakeCake^+, eatCake^+ \} \]
		Why? We can simulate this plan in both planning tasks to demonstrate it.\\
		States after $bakeCake^+$ 	
		\[S_1 = \lbrace haveCake \rightarrow 1, eatenCake \rightarrow 0, haveNoCake \rightarrow 1 \rbrace \]
		States after $eatCake^+$ 	
		\[S_2 = \lbrace haveCake \rightarrow 1, eatenCake \rightarrow 1, haveNoCake \rightarrow 1 \rbrace \]
		Goal accomplished. \\\\
		On the other hand, if we run the same plan $\pi$ in $\Pi$\\
		States after $bakeCake$ 	
		\[S_1 = \lbrace haveCake \rightarrow 1, eatenCake \rightarrow 0, haveNoCake \rightarrow 0 \rbrace \]
		States after $eatCake$ 	
		\[S_2 = \lbrace haveCake \rightarrow 0, eatenCake \rightarrow 1, haveNoCake \rightarrow 1 \rbrace \]
		The goal is not accomplished as we don't have cake.
	\end{enumerate}

	\subsection*{Exercise 6.2 - $h^+$ heuristic}
	before all we define the relaxed operator  $move^+$ where:
	\[ 
		move^+(t_m,p_{(i,j)},p_{(k,l)})=\langle 
			at(t_m,p_{(i,j)}) \land empty(p_{(k,l)}),
			at(t_m,p_{(k,l)}) \land empty(p_{(i,j)})
			\rangle
	\]
	\begin{enumerate}[label=(\alph*), listparindent=1.5em]
		\item \textbf{Show that $h^+(s) \geq h^{Mannhattan}(s)$ for 
		each legal state s of a 15-puzzle planning task}\\
			We know that $h^{Mannhattan}(s)$ is the sum of the
			distances in $i$ and $j$ to reach the goal for each tile.
			We also know that $h^+(s)$ is the minimum amount of operators
			to reach the goal state in the relaxed planning task. \\
			We 
			notice that a tile can only move to an adjacent position, 
			and that position must be empty. If a tile where to move
			to the position it would occupy in the goal state,
			assumming all the places it would have to move over
			be empty then the number of operators required for
			that tile to reach the desired position would 
			be exactly the Mannhattan distance, if its not possible
			to have a path such that the positions would be 
			empty then we would require more movements or operators
			to reach the goal state. This means that the total sum
			of operators required to reach the goal state is greater
			or equal than the sum of Mannhattan distance for each
			operator.

			\hfill\break
			(Should we mathematically formalize this type of 
			answers in the exam or an explanation like the one
			above is sufficient proof?)

			\hfill\break
			We can try to formalize this by defining the operator: 
			\[manhattanMove(tm, p_{(i,j)}, p_{(k,l)}) = \langle at(t_m, p_{(i,j)}), at(tm,p_{(k,l)})\rangle\]
			\[\text{ where } (i = k \text{ and } \abs{j - l} = 1)  \text{ or }  (j = l \text{ and } \abs{i - k}= 1) \]

			If we arrive to the goal by using these operators, then the optimal relaxed plan will be 
			the manhattan distance, furthermore we can see that the \emph{manhattanMove} dominates $move^+$
			as every plan step applicable in $move^+$ is also applicable in $manhattanMove$ as it doesn't
			have the empty precondition, as $manhattanMove$ dominates $move^+$ then
			$h^+(s) \geq h^{Mannhattan}(s)$


		\item \textbf{Show that $h^+(s) > h^{Mannhattan}(s)$ for 
		at least one state s of a 15-puzzle planning task}\\
		Consider the following state s.
		\begin{center}
			\begin{tabular}{c||c|c|c|c|}
				 i\textbackslash j & 0 & 1 & 2 & 3\\
				\hline\hline
				0 & $t_0$ & $t_1$ & $t_2$ & $t_3$ \\
				\hline
				1 & $t_4$ & $t_5$ & $t_6$ & $t_7$ \\
				\hline
				2 & $t_8$ & $t_9$ & $t_{10}$ & $t_{11}$ \\
				\hline
				3 & $t_{12}$ & $t_{14}$ & $t_{13}$ & $empty$ \\
				\hline
			\end{tabular}
		\end{center}
		To calculate the heuristic $h^+$ given 
		the state s we first consider that
		the only tiles that can move are $t_{11}$ and $t_{13}$.
		$t_{11}$ is already in its desired position but
		$t_{14}$ should be moved to $p_{(3,2)}$ and
		the precondition to do so requires that position to
		be empty first, then $t_{13}$ should move at least once
		to reach the goal state. We start by applying
		operator $move(t_{13}, p_{(2,3)}, p_{(3,3)})$ in the relaxed
		planning task, this yields the following state:

		\begin{center}
			\begin{tabular}{c||c|c|c|c|}
				 i\textbackslash j & 0 & 1 & 2 & 3\\
				\hline\hline
				0 & $t_0$ & $t_1$ & $t_2$ & $t_3$ \\
				\hline
				1 & $t_4$ & $t_5$ & $t_6$ & $t_7$ \\
				\hline
				2 & $t_8$ & $t_9$ & $t_{10}$ & $t_{11}$ \\
				\hline
				3 & $t_{12}$ & $t_{14}$ & $t_{13} \land empty $ &  $t_{13} \land empty $ \\
				\hline
			\end{tabular}
		\end{center}

		It is important to recognize that 
		because of the relaxation an state can be in several places
		at the time and that place can be ``empty" as well.
		Now we can move $t_{14}$ to its goal position by 
		applying operator $move(t_{14}, p_{(3,1)}, p_{(3,2)})$
		as the preconditions are now satisfied. This yields the state:
		
		\begin{center}
			\begin{tabular}{c||c|c|c|c|}
				 i\textbackslash j & 0 & 1 & 2 & 3\\
				\hline\hline
				0 & $t_0$ & $t_1$ & $t_2$ & $t_3$ \\
				\hline
				1 & $t_4$ & $t_5$ & $t_6$ & $t_7$ \\
				\hline
				2 & $t_8$ & $t_9$ & $t_{10}$ & $t_{11}$ \\
				\hline
				3 & $t_{12}$ & $t_{14} \land empty$ & $t_{13} \land t_{14} \land empty$ &  $t_{13} \land empty $ \\
				\hline
			\end{tabular}
		\end{center}
		
		and as $t_{13}$ already \emph{is} in $p_{(3,2)}$ we can
		simply move it to the desired position from there and
		we reach the goal state in the relaxed planning task.
		\begin{center}
			\begin{tabular}{c||c|c|c|c|}
				 i\textbackslash j & 0 & 1 & 2 & 3\\
				\hline\hline
				0 & $t_0$ & $t_1$ & $t_2$ & $t_3$ \\
				\hline
				1 & $t_4$ & $t_5$ & $t_6$ & $t_7$ \\
				\hline
				2 & $t_8$ & $t_9$ & $t_{10}$ & $t_{11}$ \\
				\hline
				3 & $t_{12}$ & $t_{13} \land t_{14} \land empty $ & $t_{13} \land t_{14} \land empty$ &  $t_{13} \land empty $ \\
				\hline
			\end{tabular}
		\end{center}

		Therefore for state s $h^+(s)=3$ as we applied only 3 operators.
		However $h^{Mannhattan}$ would have estimated 2, because for 
		$t_{14}$ the Mannhattan distance is: $\abs{3-3} + \abs{2-1}=1$,
		and for $t_{13}$: $\abs{3-3} + \abs{1-2}=1$ thus $h^{Mannhattan}(s)=2$. 
		Finally we can conclude that for this s $h^+(s) > h^{Mannhattan}(s)$
	\end{enumerate}

\end{document}